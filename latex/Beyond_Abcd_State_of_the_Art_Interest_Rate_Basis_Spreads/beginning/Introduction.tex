\chapter{Introduction}
\label{chap:intro}

{Given the ABCD interest rate basis spread model \cite{ametrano_ballabio_mazzocchi}, the aim of this work is to reparameterize it, to modify the calibration process in accordance with the following explanation and to empirically verify the methodology robustness.\footnote{In order to fully appreciated the research contents, the Author structured the work with a bottom up approach , which follows the ``medias res" Introduction, starting from the interest rate fundamentals.}
\section*{Basis time of maximum}

In the interest rate multiple curves environment, independently from the considered tenor, the market uncertainty is expected to be on the same point of time $t_{max}$.
In fact, there are no financial reasons why the market should forecast different uncertainty horizons for different tenors.
Therefore, considering the time of maximum functional form of an instantaneous abcd basis:

\begin{equation}
t_{max}(x)=\dfrac{1}{c_{x}} - \dfrac{a_{x}} {b_{x}}
\label{eqn:max_time_tenor_dependent}
\end{equation}

it is possible to rewrite it independently from the considered x tenor, making $t_{max}$ an implied constraint of the basis parameters:

\begin{equation}
t_{max}=\dfrac{1}{c_{x}} - \dfrac{a_{x}} {b_{x}}\,.
\label{eqn:max_time}
\end{equation}

The above equation shows how the $t_{max}$ needs to be the same for each tenor, thanks to the relation (tenor dependent) amongst $a$, $b$ and $c$.

A huge role is played by $c$, its value is strictly related with the maximum time, because given the maximum time functional (continuous time) form \eqref{eqn:max_time_tenor_dependent} and considering that empirically the magnitude of:

\begin{equation*}
\dfrac{1}{c_{x}}
\end{equation*}

dominates the ratio between $a_{x}$ and $b_{x}$, the value of $c$ explains a large part of the value of $t_{max}$. Therefore, in order to fix $t_{max}$ it makes sense to fix $c$.
If $c$ is fixed then \eqref{eqn:max_time} becomes:

\begin{equation}
t_{max}=\dfrac{1}{c} - \dfrac{a_{x}} {b_{x}}\,.
\label{eqn:max_time_fixed_c}
\end{equation}

This means that, in order to have the same $t_{max}$ for different tenors, there should be an implied relation between $a$ and $b$ that holds for each tenor, which for 2 generic tenor $x$ and $y$ is:

\begin{equation}
    \dfrac{a_{x}} {b_{x}}=\dfrac{a_{y}} {b_{y}}\,.
\end{equation}


Concluding, strictly related with $t_{max}$ is the corresponding value of the abcd:

\begin{equation}
s_{x}(t_{max})=\frac{b_{x}}{c_{x}} e^{(\frac{a_{x}c_{x}}{b_{x}}-1)} +d_{x} \,.
\label{eqn:max_func}
\end{equation}


\section*{Common c}

``If the exponential term is the same a difference between abcd/ABCD basis is an abcd/ABCD function". 
This is clear from a mathematical point of view indicating:
\begin{itemize}
\item $x$ as the first generic tenor;
\item $y$ as the second generic tenor, where: $x > y$;
\item $x, y$ as the difference of the above mentioned tenor: $x - y$.
\end{itemize}

It follows that, because of the tenor basis dominance explained in \cite{ametrano_ballabio_mazzocchi}, given two absolute basis:

\begin{equation*}
s_{x}(t)>s_{y}(t), \forall t > 0 
\end{equation*}

and

\begin{equation*}
s_{x}(t)-s_{y}(t), \forall t > 0
\end{equation*}

such that their respective abcd forms are:

\begin{equation*}
s_{x}(t) = (a_{x}+b_{x}t)e^{-c_{x}t}+d_{x}
\end{equation*}

and

\begin{equation*}
s_{y}(t) = (a_{y}+b_{y}t)e^{-c_{y}t}+d_{y}
\end{equation*}

if $c_{x}=c_{y}=c_{x,y}$ then the relative basis is still an abcd function:

\begin{equation}
\begin{split}
s_{x,y}(t)& =s_{x}(t)-s_{y}(t)\\
& =(a_{x}+b_{x}t)e^{-c_{x}t}+d_{x}-((a_{y}+b_{y}t)e^{-c_{y}t}+d_{y})\\
& =(a_{x}-a_{y}+(b_{x}-b_{y})t)e^{-c_{x,y}t}+d_{x}-d_{y}\\
& =(a_{x,y}+b_{x,y}t)e^{-c_{x,y}t}+d_{x,y}\,.
\end{split}
\end{equation}

Where the above reported equation is obtained using the below reported notation:

\begin{equation}
\begin{split}
a_{x,y}=a_{x}-a_{y};\\
b_{x,y}=b_{x}-b_{y} ;\\
d_{x,y}=d_{x}-d_{y} \,.\\
\label{eq:x_y_parameters}
\end{split}
\end{equation}

This result has been shown for continuous basis, but obviously it is equally valid for simple ones.


Moreover, in \cite{ametrano_ballabio_mazzocchi} has been proposed an incremental calibration where, instead of modelling the instantaneous forward rate for a given tenor x $f_{x}$ as an absolute basis on the corresponding ON instantaneous forward rate $f_{ON}$:

\begin{equation}
     f_{x}(t)=s_{x}(t)+f_{ON}(t)
     \label{eq:forward as basis + on}
\end{equation}

it is possible generalized the idea w.r.t. a general tenor y using a relative basis:

\begin{equation}
     f_{x}(t)=s_{x,y}(t)+f_{y}(t)
     \label{eq:forward as basis + y}
\end{equation}

still modelling $s_{x}(t)$ as an abcd function.

Moving forward, and considering $f_{x}$ for a generic tenor $x$, the above shown relation can be exploited.

It results that:

\begin{equation}
\begin{split}
s_{x,y}(t)& =f_{x}(t)-f_{y}(t)\\
& =s_{x}(t)+f_{ON}(t)-s_{y}(t)-f_{ON}(t)\\
& =s_{x}(t)-s_{y}(t)\\
\end{split}
\label{eq:still_abcd}
\end{equation}

is still an abcd basis, if the two basis share the common exponential term.

However, is it true that, given two generic tenors $x$ and $y$, if: 

\begin{equation}
    c_{x} \neq c_{y}
\end{equation}

then the relative basis $s_{x,y}$ is not an abcd function?
Empirical results show how it would seem that $s_{x,y}$ is still an abcd.
%\footnote{ $Abcd\_Double\_Hump\_Research.xlsx$ from the sheet $impact\_time\_dependent\_d\_lab$}%

% reductio per absurdum
Before going through algebraic calculus to proof this guess, it's preferred to proceed with a ``reductio per absurdum". If the statement: ``an abcd function shows only one $t_{max}$" is contradicted, then $s_{x,y}$ won't be an abcd function.

Given:

\begin{equation}
\begin{split}
s_{x,y}(t)& =s_{x}(t)-s_{y}(t)\\
& =(a_{x}+b_{x}t)e^{-c_{x}t}+d_{x}-((a_{y}+b_{y}t)e^{-c_{y}t}+d_{y})\\
\end{split}
\label{eq:abcd_diff_c}
\end{equation}

considering $t_{max}(x)$ equation for $s_{x}$ :

\begin{equation}
[-c_{x}(a_{x}+b_{x}t_{max}(x))+b_{x}]e^{-c_{x}t_{max}(x)}=0
\label{eqn:equation_max_time}
\end{equation}

then $t_{max}(x,y)$ equation for $s_{x,y}$ is:

\begin{equation}
\begin{split}
 [-c_{x}(a_{x}+b_{x}t_{max}(x,y))+b_{x}]e^{-c_{x}t_{max}(x,y)}-\\
 [-c_{y}(a_{y}+b_{y}t_{max}(x,y))+b_{y}]e^{-c_{y}t_{max}(x,y)}=0\,.
\label{eq:t_max_relative}
\end{split}
\end{equation}

Unfortunately, it is impossible to retrieve $t_{max}(x,y)$ because of the problematic form:

\begin{equation}
    y=b e^{b}
\label{eq:problem}
\end{equation}

Because of this result it is impossible to study the problem with a ``reductio per absurdum" and it is necessary to make an attempt retrieving an abcd from $s_{x,y}$:

\begin{equation}
\begin{split}
s_{x,y}(t)& =s_{x}(t)-s_{y}(t)\\
& =(a_{x}+b_{x}t)e^{-c_{x}t}+d_{x}-((a_{y}+b_{y}t)e^{-c_{y}t}+d_{y})\\
&=(a_{x}+b_{x}t)e^{-c_{x}t}+d_{x,y}\,,
\end{split}
\label{eq:s_x,y_is_abcd}
\end{equation}

where: $d_{x,y}=d_{x}-((a_{y}+b_{y}t)e^{-c_{y}t}+d_{y})$\,.


This means that $s_{x,y}$ is still an abcd basis, but $d$ is time dependent, reason why looking at the empirical results it seemed an abcd one.
Therefore, it makes no sense to talk about relative abcd retrieved from absolute abcd with different $c$.

Given these findings, what did it happen when $s_{x,y}$ was treated as an abcd in \cite{ametrano_ballabio_mazzocchi}? 
They calibrated $s_{6M}$ on the bootstrapped ON curve and then they obtained by construction $s_{x,6M}$ as abcd. Given that $s_{6M}$ and $s_{x,6M}$ were abcd, but with different $c$, it follows that $s_{x}$ was not abcd.
The only drawback in the adopted scheme is the impossibility to swap from relative to absolute basis.
Acting in this way \cite{ametrano_ballabio_mazzocchi} created two different models, for the absolute basis and for the relative ones, that can be reconciled guaranteeing that $c$ is the same for both the absolute and relative basis. 
Obviously, according to the performed calibration (incremental or not), different parameters are estimated, but with a common $c$ is possible to swap from a basis to the other independently from the calibration type.


\section*{Aim}

For all the above mentioned reasons, the ideas that need to be investigated are:

\begin{enumerate}

\item reparameterization of the model, in its continuous form, relying on the following parameters: 

\begin{equation}
    a_{x}, d_{x}, t_{max}, s_{x}(t_{max})\,.
    \label{parameters}
\end{equation}

The new parametric form gives financial meaning to the model parameters, because:

\begin{itemize}
    \item $a_{x} + d_{x}$ is the value of the abcd corresponding to $t=0$;
    \item $d_{x}$ is the long run value of the abcd basis;
    \item $t_{max}$ represents the peak of uncertainty in terms of time;
    \item $s_{x}(t_{max})$ is the value of the abcd at the peak of uncertainty.
\end{itemize}
Remarkable feature should be that, after retrieving the new parametric form, all the analytical findings from \cite{ametrano_ballabio_mazzocchi} could be exploited, just moving from the new parametric form to the old abcd one.
In this way, the users are given an interface, which allows them to directly manage the maxima of the basis and the basis time of maximum as observed on the markets.

\item modification of the framework according to the idea of a globally shared $t_{max}$, studying the financial intuition behind it;

\item modification of the framework according to the idea of a globally shared $t_{max}$ and $c$, in order to completely validate the model and the underling financial intuition.

\end{enumerate}
}