\chapter*{Acknowledgements}

Life is a set of time windows, each window has a certain level of consumable energy which a person can address as she prefers.
People are ``containers of ordered necessities" which need to be filled: ``You cannot sleep well if you have not eaten well, you cannot eat well if you cannot afford a good meal...".
Studying is not at the top of the ``what you need to do" list and in order to stay focused you need both to manage yourself and your surrounding, but the most of time you cannot manage the environment where you grow and poor relationships can really drain your lymph.\\

Fortunately, during the last 5 years thanks to my choices and my incredible native luck, I have been surrounded by positive people who really helped me developing myself and serenely refine my skills.
I would like to thank all my Family members, especially my Parents, because they supported me with all their means and accomplished the vital functions allowing me to save energy without ``third parties problems".
I would like to thank Miss Gemma Cortinovis because we have really built and we are continuously building an extremely sound anti-fragile relationship which allows me to recreate energy from the scratch and forget about the boring life necessity list.
In the end, I would like to thank my Friends, few but all bad boys, and my university fellows who really helped me to enjoy the time at school and acquire knowledge.\\

From a technical point of view all my thanks go to Prof. Ferdinando Ametrano who I really appreciated because of his will to stimulate an endearing intellectual debate about quantitative finance and its life choices implications, Paolo Mazzocchi who gave me a friendly help to overcome crude technical problems, Luigi Ballabio and all QuantLib developers for their legacy and the Deloitte and Banca IMI colleagues, both from Product Control and Financial Engineering offices, for giving me insights which allowed me to produce a real world research.
Concluding, thanks to the University of Milano-Bicocca and its professors, especially Prof. Paola Bongini, for shaping my mind during these years.
